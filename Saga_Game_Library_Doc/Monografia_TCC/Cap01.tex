% ----------------------------------------------------------------
% Introduo e Conceitos Básicos *******************
% ----------------------------------------------------------------
\chapter{Introdução}
\label{cap:indtroducao}
%
Desde os primórdios da humanidade a competição é uma forma de diversão muito popular. O objetivo de qualquer competição é testar uma habilidade individual ou grupal e destacar quem ganha. Embora essa característica ainda seja a mesma, a competição está condicionada a uma evolução que pode ser notada, por exemplo, na corrida: no começo ela era individual e só testava a velocidade e a resistência da pessoa; hoje uma corrida automobilística testa a resistência, a destreza, a inteligência e a tecnologia da equipe.
Mas essa evolução não se da só na complexidade da competição, ela se manifesta da mesma forma na sua abstração. Os jogos de estratégia que nós conhecemos hoje, por exemplo, são versões abstratas das competições físicas. A aptidão física de cada criatura imaginária é determinada por alguma característica interna do jogo, porque o que o jogo de estratégia testa é o raciocínio e a estratégia da pessoa e não a sua capacidade física propriamente dita. O xadrez internacional, simulando uma guerra da Idade Média, é um caso concreto dessa evolução.
Com o avanço da microeletrônica e da computação, no entanto, o jogo de estratégia ganha uma plataforma que pode simular não só um sistema de lógica, mas toda uma realidade virtual. Qualquer jogo pode ganhar uma versão eletrônica. O jogo eletrônico, portanto, não é só uma brincadeira de criança, ele é na verdade o último estágio de um passatempo milenar. Isso se confirma pela movimentação de recursos e ganhos da indústria dos jogos eletrônicos desta geração.
%
%
% ----------------------------------------------------------------
% Motivacao *******************
% ----------------------------------------------------------------
\section{Motivação}
\label{section:motivacao}
%
%
%
Nos últimos anos, o mercado de games do Brasil tem presenciado um crescimento significativo. \cite{e}
O advento dos dispositivos móveis, como \textit{smartphones} e \textit{tablets}, e a possibilidade de comercializar seu produto \textit{online} em lojas virtuais e assim reduzir custos favoreceu, em parte, a redução da pirataria e fez com que o usuário preferisse a compra do produto original a investir em um produto não-original. Essa mudança de comportamento por parte do consumidor fez com que um mercado, que antes era visto como inseguro, passasse a ser considerado um mercado promissor pelas empresas desenvolvedoras de software, incluindo as desenvolvedoras de \textit{games}.
\par
Com o crescimento da área de desenvolvimento de jogos eletrônicos, surge também a necessidade de encontrar mão-de-obra capacitada, necessidade esta que é uma das maiores reclamações das indústrias de desenvolvimento de \textit{games} do Brasil. Por se tratar de uma área de desenvolvimento recente, é difícil encontrar profissionais capacitados na área. Uma das soluções mais simples para esta carência de mão-de-obra é incentivar estudantes, sejam eles de nível técnico ou universitário, a aprender sobre as ferramentas e técnicas mais utilizadas no desenvolvimento de um jogo eletrônico. Assim, torna-se de suma importância a implementação de ferramentas que facilitem o primeiro contato do estudante com essa complexa área de desenvolvimento, motivo este que nos motivou a criaçãoo da \textit{SAGA Game Library}.
%
%
% ----------------------------------------------------------------
% Objetivos *******************
% ----------------------------------------------------------------
\section{Objetivos}
\label{section:objetivos}
%
A \textit{SAGA Game Library} foi desenvolvida tendo em foco o meio acadêmico. Com o aumento do mercado de desenvolvimento de 
jogos eletrônicos no país, surge a necessidade de investir na capacitação de profissionais para atender a essa demanda. Não apenas profissionais do setor precisam estar em constante atualização, mas os agora estudantes e futuros profissionais também precisam de capacitação. É para esse último que é direcionada esta biblioteca de desenvolvimento. Seu objetivo primário é possibilitar ao usuário, seja ele um estudante ou entusiasta, o primeiro contato com o mundo do desenvolvimento de jogos.
%
\par
%
É certo que já existem muitas \textit{game engines}, inclusive em C++, mas o estudo é o piso de todas as descobertas científicas, o que justifica e motiva o desenvolvimento de uma biblioteca de jogos didática. Esta é a nossa proposta: uma camada de orientação a objetos envolvendo a Allegro de uma forma simples e didática. Simplicidade, eficiência e aprendizado são as palavras-chave da \textit{SAGA Game Library}.
%
%
%