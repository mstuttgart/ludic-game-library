% ----------------------------------------------------------------
% Revisão bibliográfica *******************
% ----------------------------------------------------------------
\chapter{Desenvolvimento}
\label{cap:desenvolvimento}
%
%
O desenvolvimento da \textit{SAGA Game Library} ou SGL, assim como todo \textit{software}, passou por diversas etapas para que no final torna-se possível obter um produto condizente com a proposta do trabalho.
\par 
O desenvolvimento de um \textit{software}, de maneira geral, sempre é composto das seguintes etapas:
%
\begin{itemize}
 \item Especificação dos requisitos do \textit{software}: Descrição do objetivo e do se espera do \textit{software}.
 \item Projeto do sistema: decisão dos conceitos relacionado ao que deve ser implementado, incluindo a escolha da linguagem de programação
 adequada, sistema operacional alvo, bibliotecas e ferramentas auxiliares.
 \item Implementação: O próprio desenvolvimento do \textit{software}. Consiste na transformação de todo conteúdo formulado na fase de projeto em código.
 \item Teste e depuração: Fase que consiste no teste do \textit{software} já implementado e procura por erros e correção destes.
 \item Documentação: A fase final do desenvolvimento consiste em documentar o \textit{software} criado, incluindo manuais de uso da biblioteca.
\end{itemize}
%
\par 
A SGL também seguiu de maneira consiste as etapas acima e a seguir serão descritas as particularidades de cada uma delas.
%
%