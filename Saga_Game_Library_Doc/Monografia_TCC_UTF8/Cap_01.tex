% ----------------------------------------------------------------
% Introdução e Conceitos Básicos *******************
% ----------------------------------------------------------------
\chapter{Introdução}
\label{cap:indtroducao}
%
Desde os primórdios da humanidade a competição é uma forma de diversão muito popular. O objetivo de qualquer competição é testar uma habilidade individual ou grupal e destacar quem ganha. Embora essa característica ainda seja a mesma, a competição está condicionada a uma evolução que pode ser notada, por exemplo, na corrida: no começo ela era individual e só testava a velocidade e a resistência da pessoa; hoje uma corrida automobilística testa a resistência, a destreza, a inteligência e a tecnologia da equipe.
Mas essa evolução não se da só na complexidade da competição, ela se manifesta da mesma forma na sua abstração. Os jogos de estratégia que nós conhecemos hoje, por exemplo, são versões abstratas das competições físicas. A aptidão física de cada criatura imaginária é determinada por alguma característica interna do jogo, porque o que o jogo de estratégia testa é o raciocínio e a estratégia da pessoa e não a sua capacidade física propriamente dita. O xadrez internacional, simulando uma guerra da Idade Média, é um caso concreto dessa evolução.
Com o avanço da microeletrônica e da computação, no entanto, o jogo de estratégia ganha uma plataforma que pode simular não só um sistema de lógica, mas toda uma realidade virtual. Qualquer jogo pode ganhar uma versão eletrônica. O jogo eletrônico, portanto, não é só uma brincadeira de criança, ele é na verdade o último estágio de um passatempo milenar. Isso se confirma pela movimentação de recursos e ganhos da indústria dos jogos eletrônicos desta geração.
%
%
% ----------------------------------------------------------------
% Motivacao *******************
% ----------------------------------------------------------------
\section{Motivação}
%
%
%
\subsection{Mercado de \textit{games}: Panorama mundial}
%
Atualmente o mercado de games é consolidado como um dos principais dentro do segmento de entretenimento, junto das indústrias cinematográfica, fonográfica e literária. Em outros ramos, estando atrás apenas das indústrias bélica e de automóveis. Além disso, é o que cresce mais rapidamente, à medida que o hardware utilizado se torna mais poderoso, uma nova geração de software é desenvolvida, causando uma experiência nova de interação do jogador. 
Esse crescimento é de fato acelerado, pois de um modo geral a performance dos processadores dobra a cada 12 a 18 meses. Outros fatores importantes são as mudanças no desenvolvimento dos games que passaram a ter grandes investimentos, e também o crescimento do público-alvo influenciado pelos meios de comunicação, tendo os Estados Unidos como o principal mercado consumidor, com 39%.
Mas nem sempre foi assim, o mercado de games já foi precário e desanimador. Os programadores não eram creditados pelo desenvolvimento dos games e tinham pouco tempo para essa tarefa, além de um ambiente de pressão estressante para finalizar o game. Como consequência, o número de jogos malfeitos era enorme. 
%
%
\subsection{Brasil – problemas enfrentados nos últimos dez anos}
%
No cenário brasileiro, o mercado de games comporta-se timidamente em relação ao panorama internacional. Há controvérsias em relação a esse assunto. Para alguns o mercado está ainda em formação, e para outros, já está implantado e até certo ponto consolidado, mas necessita de ajustes.
Há pouquíssimo interesse das empresas internacionais se instalarem no Brasil. Os motivos passam pela estrutura do mercado onde a pirataria tem grande influência. Em 2006, o número de produtos piratas representava impressionantes 90\% dos jogos comercializados. Outro agravante é a alta tributação, causando a desestruturação do mercado interno e não justificada uma vez que não existem hardwares de videogames de última geração sendo produzidos no Brasil. 
O setor tem carência de profissionais mais especializados e experientes. A maioria das empresas é muito jovem. Os profissionais brasileiros mais experientes preferem trabalhar fora do país. Conhecidos pela técnica e profissionalismo, além da paixão pelos jogos, eles acabam sendo contratados por empresas fora do País e muitas vezes não voltam.
Segundo Davi Nakano - professor Doutor da Universidade de São Paulo - uma maneira de ajudar o setor é criando demanda, que poderia vir, por exemplo, com a valorização de jogos para o ensino, nas escolas.
%
\subsection{Brasil – o que mudou nos últimos cinco anos}
%
Apesar de tantos problemas, nosso país é um mercado em potencial. Temos tudo para ser uma força na indústria de jogos. O mercado está em ritmo acelerado de crescimento e os videogames são a principal forma de entretenimento para os brasileiros de todas as idades. De acordo com o Banco Nacional de Desenvolvimento Econômico e Social (BNDES), o setor de jogos eletrônicos nacional ainda precisa melhorar, principalmente no que diz respeito a incentivo às empresas de games a criarem mais propriedades intelectuais que possam virar franquias e serem comercializadas tanto aqui quanto no exterior. E ainda, melhorar a qualidade profissional na formação e oferta de empregos.
De fato, estão sendo feitos esforços para tornar o país mais amigável em relação a essa questão. Nos últimos anos, o governo brasileiro abriu os olhos e está dando mais atenção à indústria de games. Surgiram parcerias com empresas internacionais e também com universidades para dar mais suporte e condições ao recém-formado para futuras vagas no mercado de trabalho. Também a implementação de políticas públicas, para que haja condições de desenvolvimento e proteção aos produtos nacionais. Hoje contamos com 26 universidades com cursos que envolvem jogos, e o público alvo do nosso mercado tem mais de 45 milhões de usuários. Em jogos educativos, o Brasil já é uma referência mundial.
As empresas especializadas no ramo têm se multiplicado no País. Segundo estudo da Associação Brasileira dos Desenvolvedores de Games (Abragames) existem 220 empresas no país. Há pouco tempo atrás, em 2008, eram somente 48.
Pesquisa da consultoria holandesa Newszoo diz que nos próximos dois anos, o mercado de games deverá crescer no Brasil o dobro do esperado para a média mundial, à uma taxa de 15\% ao ano
Já somos o quarto maior consumidor de jogos eletrônicos, atrás apenas de Ásia, Europa e Estados Unidos. Em contradição com a maioria dos países, onde o setor tem encolhido, o Brasil é o país onde o mercado de jogos eletrônicos mais cresceu em 2012, sendo que houve um aumento de 60\% se comparado a 2011. Além disso, São Paulo é o maior mercado consumidor de games originais do mundo.
Em cerca de 10 anos reduzimos a pirataria de 90\% para próximo de 50\% , e na América Latina já somos o país com o melhor desempenho nesse combate. Fato é que há muito ainda de jogos piratas a diminuir, mas essa redução já tem um impacto extremamente positivo no mercado de games. Cada dólar investido em software oficial injeta 437 dólares no setor.
Recentemente, duas fábricas de peso se instalaram aqui: XBOX e PlayStation 3. A primeira só existia na China e a segunda além de existir na China, tem filiais também no Japão e em um país do leste Europeu. Isso demonstra a importância do nosso mercado e gera empregos e facilidade de acesso para o consumidor final.

Um fenômeno recente também é a troca de geração dos consoles. Por um lado, tem-se o compartilhamento de conteúdo diretamente nas redes sociais e a redução do consumo de mídia física, já que a mídia digital está ganhando cada vez mais espaço. Por outro lado os games para tablet, celulares e jogos online devem continuar crescendo por serem opções de baixo custo e que rodam em dispositivos que o usuário já possui. Esses fatores inevitavelmente ajudam no combate a pirataria.
%
%

Nos últimos anos, o mercado de games do Brasil tem presenciado um crescimento significativo. 
O advento dos dispositivos móveis, como \textit{smartphones} e \textit{tablets}, e a possibilidade de comercializar seu produto \textit{online} em lojas virtuais e assim reduzir custos favoreceu, em parte, a redução da pirataria e fez com que o usuário preferisse a compra do produto original a investir em um produto não-original. Essa mudança de comportamento por parte do consumidor fez com que um mercado, que antes era visto como inseguro, passasse a ser considerado um mercado promissor pelas empresas desenvolvedoras de software, incluindo as desenvolvedoras de \textit{games}.
\par
Com o crescimento da área de desenvolvimento de jogos eletrônicos, surge também a necessidade de encontrar mão-de-obra capacitada, necessidade esta que é uma das maiores reclamações das indústrias de desenvolvimento de \textit{games} do Brasil. Por se tratar de uma área de desenvolvimento recente, é difícil encontrar profissionais capacitados na área. Uma das soluções mais simples para esta carência de mão-de-obra é incentivar estudantes, sejam eles de nível técnico ou universitário, a aprender sobre as ferramentas e técnicas mais utilizadas no desenvolvimento de um jogo eletrônico. Assim, torna-se de suma importância a implementação de ferramentas que facilitem o primeiro contato do estudante com essa complexa área de desenvolvimento, motivo este que nos motivou a criação da \textit{SAGA Game Library}.
%
%
% ----------------------------------------------------------------
% Objetivos *******************
% ----------------------------------------------------------------
\section{Objetivos}
\label{section:objetivos}
%
A \textit{SAGA Game Library} foi desenvolvida tendo em foco o meio acadêmico. Com o aumento do mercado de desenvolvimento de 
jogos eletrônicos no país, surge a necessidade de investir na capacitação de profissionais para atender a essa demanda. Não apenas profissionais do setor precisam estar em constante atualização, mas os agora estudantes e futuros profissionais também precisam de capacitação. É para esse último que é direcionada esta biblioteca de desenvolvimento. Seu objetivo primário é possibilitar ao usuário, seja ele um estudante ou entusiasta, o primeiro contato com o mundo do desenvolvimento de jogos.
%
\par
%
É certo que já existem muitas \textit{game engines}, inclusive em C++, mas o estudo é o piso de todas as descobertas científicas, o que justifica e motiva o desenvolvimento de uma biblioteca de jogos didática. Esta é a nossa proposta: uma camada de orientação a objetos envolvendo a Allegro de uma forma simples e didática. Simplicidade, eficiência e aprendizado são as palavras-chave da \textit{SAGA Game Library}.
%
%
%