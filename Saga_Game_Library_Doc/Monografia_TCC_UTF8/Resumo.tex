\begin{resumo}

Uma biblioteca de jogos ou \textit{game engine} pode ser vista como um conjunto de recursos e ferramentas para a construção de um jogo. Você pode criar um jogo sem uma biblioteca básica, assim como você pode criar uma mesa de madeira sem pregos, martelos, parafusos, chaves de fenda e serras, mas as vantagens que as ferramentas proporcionam justificam chamá-las de necessárias.
O nível dessas ferramentas varia: algumas \textit{engines} se limitam a códigos, ou seja, constantes, variáveis, funções e classes relacionadas, mas outras contam com interfaces gráficas que possibilitam o desenvolvimento de um jogo sem codificação alguma. De qualquer forma, uma \textit{game engine} precisa proporcionar, no mínimo, ferramentas para manipular sons, imagens (texto, imagens, etc), memória (dados) e controle (teclado, mouse, etc).

\vspace{1em}
\textbf{Palavras-chave}: \textit{game engine}, jogos eletrônicos, ferramentas de desenvolvimento.
\end{resumo}

\begin{resumo}[Abstract]

A game engine is a set of game development resources and tools. One may create a game without a base engine, just like one can create a wooden table without nails, hammers, screws, screwdrivers and saws, but the advantages tools provide legitimate calling them necessary.
These tools’ level vary: some engines are only about code, i.e. constants, variables, functions and related classes, but others come with graphic interfaces that enable the development of a game without any coding. Anyway, a game engine must provide, at least, sound, images (text, images, etc), memory (data) and control (keyboard, mouse, etc) manipulation tools.

\vspace{1em}
\textbf{Keywords}: game engine, electronic games, development tools.
\end{resumo}