\chapter{Conclus�o}
\label{cap:conclusao}

Atrav�s deste projeto, n�s conclu�mos que o desenvolvimento de jogos eletr�nicos, em toda a sua extens�o, envolve v�rias habilidades importantes para qualquer profiss�o da �rea da tecnologia, como a criatividade e a l�gica matem�tica, especialmente para a tecnologia da informa��o e a engenharia da computa��o, como as t�cnicas de programa��o e a engenharia de software, o que confirma a viabilidade do uso dessa atividade de uma forma did�tica.

Ainda mais considerando a tend�ncia da inclus�o de disciplinas de programa��o na grade do ensino fundamental \nocite{OLHARDIG}, em qual contexto essa atividade seria um grande incentivo, sen�o no m�nimo um meio de divulga��o da programa��o, que por sua vez colaboraria com a educa��o de uma forma geral. Por mais ut�pica que pare�a, essa expectativa est� de acordo com a nossa realidade -- o Brasil � o quarto maior consumidor de games do mundo \nocite{ESTADAO}. 
\par 
O projeto nos proporcionou um rico aprendizado sobre a �rea de desenvolvimento de jogos eletr�nicos, bem como das t�cnicas utilizadas no desenvolvimento e na programa��o dos mesmos. Por meio de pesquisa e dedica��o, conseguimos implementar um \textit{software} que foi ao encontro de todos os requisitos estipulados no inicio do projeto. Uma vez que a \textit{SAGA Game Library} esteja finalizada e totalmente testada, pretendemos torn�-la um projeto de \textit{software} livre, de modo que tanto escolas e estudantes quanto entusiastas possam utiliz�-la para projetos educacionais, pessoais ou at� mesmo comerciais.