\chapter{Conclusão}
\label{cap:conclusao}

Através deste projeto, nós concluímos que o desenvolvimento de jogos eletrônicos, em toda a sua extensão, envolve várias habilidades importantes para qualquer profissão da área da tecnologia, como a criatividade e a lógica matemática, especialmente para a tecnologia da informação e a engenharia da computação, como as técnicas de programação e a engenharia de software, o que confirma a viabilidade do uso dessa atividade de uma forma didática.
\par
Ainda mais considerando a tendência da inclusão de disciplinas de programação na grade do ensino fundamental \nocite{OLHARDIG}, em qual contexto essa atividade seria um grande incentivo, senão no mínimo um meio de divulgação da programação, que por sua vez colaboraria com a educação de uma forma geral. Por mais utópica que pareça, essa expectativa está de acordo com a nossa realidade -- o Brasil é o quarto maior consumidor de games do mundo \nocite{ESTADAO}. 
\par 
O projeto nos proporcionou um rico aprendizado sobre a área de desenvolvimento de jogos eletrônicos, bem como das técnicas utilizadas no desenvolvimento e na programação dos mesmos. Por meio de pesquisa e dedicação, conseguimos implementar um \textit{software} que foi ao encontro de todos os requisitos estipulados no inicio do projeto, ou seja, uma \textit{game engine} simples de usar, até mesmo para iniciantes na programação, incluindo características como portabilidade do código e uma ferramenta para facilitar o gerenciamento dos recursos dos jogos, que somam opções e qualidades sem no entanto diminuir a usabilidade da mesma. Usabilidade pode ser averiguada quando se observa os exemplos de uso da \textit{SAGA Game Library} disponíveis neste trabalho.
\par
Uma vez que a \textit{SAGA Game Library} esteja finalizada e totalmente testada, pretendemos torná-la um projeto de \textit{software} livre, de modo que tanto estudantes e escolas quanto entusiastas possam utilizá-la para projetos pessoais, educacionais ou até mesmo comerciais.