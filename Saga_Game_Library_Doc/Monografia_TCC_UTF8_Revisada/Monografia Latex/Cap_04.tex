% ----------------------------------------------------------------
% Motivacao *******************
% ----------------------------------------------------------------
\chapter{Resultados}
\label{cap:resultados}
%
%
%
%Nos últimos anos, o mercado de games do Brasil tem presenciado um crescimento significativo. 
%O advento dos dispositivos móveis, como \textit{smartphones} e \textit{tablets}, e a possibilidade de comercializar seu produto \textit{online} em lojas virtuais e assim reduzir custos favoreceu, em parte, a redução da pirataria e fez com que o usuário preferisse a compra do produto original a investir em um produto não-original. Essa mudança de comportamento por parte do consumidor fez com que um mercado, que antes era visto como inseguro, passasse a ser considerado um mercado promissor pelas empresas desenvolvedoras de software, incluindo as desenvolvedoras de \textit{games}.
%\par
%Com o crescimento da área de desenvolvimento de jogos eletrônicos, surge também a necessidade de encontrar mão-de-obra capacitada, necessidade esta que é uma das maiores reclamações das indústrias de desenvolvimento de \textit{games} do Brasil. Por se tratar de uma área de desenvolvimento recente, é difícil encontrar profissionais capacitados na área. Uma das soluções mais simples para esta carência de mão-de-obra é incentivar estudantes, sejam eles de nível técnico ou universitário, a aprender sobre as ferramentas e técnicas mais utilizadas no desenvolvimento de um jogo eletrônico. Assim, torna-se de suma importância a implementação de ferramentas que facilitem o primeiro contato do estudante com essa complexa área de desenvolvimento, motivo este que nos motivou a criação da \textit{SAGA Game Library}.
%
%