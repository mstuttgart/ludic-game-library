\begin{resumo}

Uma biblioteca para jogos ou \textit{game engine} pode ser vista como um conjunto de recursos e ferramentas para a construção de um jogo. Você pode criar um jogo sem uma biblioteca básica, assim como você pode criar uma mesa de madeira sem martelos, chaves de fenda e serras, mas as vantagens que essas ferramentas proporcionam justificam chamá-las de necessárias.

O nível dessas ferramentas varia: algumas \textit{engines} se limitam a códigos, ou seja, constantes, variáveis, funções e classes relacionadas, mas outras contam com interfaces gráficas que possibilitam o desenvolvimento de um jogo sem codificação alguma. De qualquer forma, uma \textit{game engine} pode proporcionar ferramentas para manipular sons, imagens (texto, imagens, etc), memória (dados) e controle (teclado, mouse, etc), entre outras.

\textit{SAGA Game Library} é uma \textit{engine} voltada para estudantes da programação, especialmente a programação de jogos, pelo que foi feita priorizando a simplicidade a qualquer outra característica, usando a biblioteca ALLEGRO para acessar as funções básicas do sistema. Essa \textit{engine} oferece algumas ferramentas básicas para o desenvolvimento de um jogo 2D, à medida em que essas não concorrem com a simplicidade da mesma.

\vspace{1em}
\textbf{Palavras-chave}: \textit{game engine}, jogos eletrônicos, ferramentas de desenvolvimento.
\end{resumo}

\begin{abstract}
%
A game engine is a set of game development resources and tools. One may create a game without a base engine, just like one can create a wooden table without hammers, screwdrivers and saws, but the advantages tools provide legitimate calling them necessary.

These tools’ level vary: some engines are only about code, i.e. constants, variables, functions and related classes, but others come with graphic interfaces that enable the development of a game without any coding. Anyway, a game engine may provide sound, images (text, images, etc), memory (data) and control (keyboard, mouse, etc) manipulation tools, as others.

SAGA Game Library is an engine oriented to students of programming, specially game programming, which is why it was made focusing simplicity instead of any other feature, using ALLEGRO library to access the system's basic functions. This engine offers some game development basic tools, as long as these don't compete with its simplicity.

\vspace{2em}
\textbf{Keywords}: game engine, electronic games, development tools.
\end{abstract}